\documentclass[assignment2.tex]{subfiles}
\begin{document}

\section*{5η Άσκηση}
Για την επίλυση αυτής της άσκησης, υλοποιήθηκε ο αλγόριθμος απαλοιφής \textlatin{Gauss} με ένα επιπλέον όρισμα, το οποίο καθορίζει το είδος της οδήγησης που θα εφαρμοστεί. 

Το όρισμα είναι προαιρετικό οπότε σε περίπτωση που δεν δοθεί, εκτελείται ο αλγόριθμος χωρίς οδήγηση, παρά μόνο για έλεγχο μη μηδενικού στοιχείου οδηγού.

Η ορίζουσα του δοσμένου πίνακα μπορεί να υπολογιστεί εύκολα αφού εκτελεστεί ο αλγόριθμος απαλοιφής. Μετά την εκτέλεση του αλγορίθμου, ο αρχικός πίνακας είναι άνω τριγωνικός και επομένως η ορίζουσά του είναι το γινόμενο των διαγώνιων στοιχείων.

Τελικά είναι $det(A)=-8$ είτε εφαρμοστεί μερική είτε πλήρης οδήγηση. Η λύση με μερική οδήγηση είναι
\begin{equation}
x_{pp} = \left[
\begin{matrix}
3 \\ 2 \\ 1
\end{matrix}
\right]
\end{equation}
και για πλήρη οδήγηση
\begin{equation}
x_{fp} = \left[
\begin{matrix}
1 \\ 3 \\ 2
\end{matrix}
\right]
\end{equation}

Η διαφορά στη λύση οφείλεται στις μεταθέσεις γραμμών που προέκυψαν κατά την εκτέλεση του αλγορίθμου.

Παρακάτω ακολουθεί ο κώδικας που γράφτηκε σε \textlatin{Python} και έγινε χρήση της βιβλιοθήκης \textlatin{Numpy}. Η υλοποίηση του αλγορίθμου απαλοιφής \textlatin{Gauss} δίνεται στο Παράρτημα.
\selectlanguage{english}
\lstinputlisting[style=python, firstline=8]{ex5.py}
\end{document}