\usepackage{amsmath}                % Mathematical Notation
\usepackage{amssymb}                % Mathematical Symbols
\usepackage[english,greek]{babel}   % Multilingual support for Plain TEX or LATEX
\usepackage{color}                  % Used for defining colors
\usepackage[b]{esvect}              % Vector Arrows
\usepackage{graphicx}               % Enhanced support for graphics
%\usepackage{greektex}
\usepackage[unicode]{hyperref}      % URL support
\usepackage{listings}               % Source Code Listings
\usepackage{placeins}               % Float barrier
\usepackage[utf8x]{inputenc}        % Accept different input encodings
\usepackage{setspace}               % Set space between lines
\usepackage{subfiles}               % Individual typesetting of subfiles of a "main" document 

\renewcommand{\vec}[1]{\vv{#1}}
\newcommand{\dvec}[1]{\dot{\vv{#1}}}
\newcommand{\ddvec}[1]{\ddot{\vv{#1}}}
\newcommand{\derivative}[3]{\displaystyle\left.\frac{d#1}{d#2}\right|_{#3}}
\newcommand{\nderivative}[4]{\displaystyle\left.\frac{d^{#1}#2}{d#3^{#1}}\right|_{#4}}

\onehalfspacing

\definecolor{mygreen}{rgb}{0,0.6,0}
\definecolor{mygray}{rgb}{0.5,0.5,0.5}
\definecolor{mymauve}{rgb}{0.58,0,0.82}

\lstdefinestyle{python}{ %
    basicstyle=\footnotesize,
    breakatwhitespace=false,
    breaklines=true,
    commentstyle=\color{mygreen}\ttfamily,
    deletekeywords={sum},
    frame=single,
    keepspaces=true,
    keywordstyle=\color{blue},
    language=Python,
    morekeywords={as, print},
    numbers=left,
    numberstyle=\tiny\color{mygray},
    showstringspaces=false,
    stepnumber=5,
    stringstyle=\color{mymauve},
    tabsize = 1,
    title=\lstname
}