\documentclass[assignment2.tex]{subfiles}
\begin{document}

\section*{6η Άσκηση}
Σε αυτή την άσκηση, είναι συνετό πριν εφαρμοστεί οποιοδήποτε επαναληπτικό σχήμα να εκτελεστούν κάποιες αλγεβρικές πράξεις. Ο στόχος είναι το σχήμα να μπορεί να προγραμματιστεί πιο εύκολα σε υπολογιστή και ενδεχομένως να αποφευχθούν επαναλήψεις υπολογισμών.
\begin{equation}
\begin{split}
x^{k+1} &= (1-\tau)x^k + Ux^{k+1}+(\tau-1)Ux^k + \tau Lx^k + \tau D^{-1}d \rightarrow \\
(I-U)x^{k+1} &= \left[(1-\tau)I - (1-\tau)U\right]x^k +\tau Lx^k + \tau D^{-1}d \rightarrow \\
x^{k+1} &= (1-\tau)Ix^k + \tau (I-U)^{-1}Lx^k + \tau (I-U)^{-1}D^{-1}d \rightarrow \\
x^{k+1} &= \left[(1-\tau)I + \tau (I-U)^{-1}L\right]x^k + \tau (I-U)^{-1}D^{-1}d \xrightarrow{M=I-U}\\
x^{k+1} &= G x^k + c
\end{split}
\label{eq:iterative_scheme}
\end{equation}
όπου $G=(1-\tau)I + \tau M^{-1}L$, $c=\tau M^{-1}D^{-1}d$. Όπως φαίνεται από την εξίσωση (\ref{eq:iterative_scheme}), τα $G$ και $c$ είναι σταθερά και δεν αλλάζουν σε κάθε επανάληψη. Επομένως μπορούν να υπολογιστούν εξαρχής και απλά να χρησιμοποιούνται στο σώμα του υπολογιστικού βρόχου.

Επίσης, σχετικά με τον πίνακα $M=(I-U)$, από τον ορισμό του επαναληπτικού σχήματος, ο πίνακας $U$ είναι αυστηρά άνω τριγωνικός, ο $I$ είναι ο μοναδιαίος πίνακας, άρα η διαφορά τους θα είναι άνω τριγωνικός πίνακας, με μοναδιαία διαγώνια στοιχεία.

Ο υπολογισμός του αντιστρόφου είναι πλέον εύκολη υπόθεση αφού μπορεί να υπολογιστεί η \textlatin{j}-στήλη του $M^{-1}$, ως η λύση ενός συστήματος $M c_j = e_j$, όπου $e_j$ η \textlatin{j}-στήλη του μοναδιαίου πίνακα. Η επίλυση του τελευταίου συστήματος γίνεται με πίσω αντικατάσταση, αξιοποιώντας την ιδιότητα του $M$ να είναι άνω τριγωνικός.

Αντίστοιχα εύκολος είναι και ο υπολογισμός του $D^{-1}$. Ο $D$ είναι διαγώνιος πίνακας και δεδομένου ότι κανένα στοιχείο της διαγωνίου δεν είναι 0, ο αντίστροφός του είναι πάλι ένας διαγώνιος πίνακας μου τα διαγώνια στοιχεία τα αντίστροφα του $D$. Αξιοποιώντας τα παραπάνω, υπολογίζονται εκτός επαναληπτικού βρόχου $D^{-1}$, $L$, $U$, $(I-U)^{-1}$, $G$ και τέλος $c$.

Για την επίδειξη της συμπεριφοράς του αλγορίθμου, χρησιμοποιήθηκαν ένας τριδιαγώνιος πίνακας $4x4$ και ένα διάνυσμα $d$. Ο αλγόριθμος εκτελέστηκε για $\tau=0.1, 0.2, \dots, 1.9$ με σκοπό να μελετηθεί η συμπεριφορά του επαναληπτικού σχήματος.

Για τα συγκεκριμένα $A$, $d$ ο αλγόριθμος συνέκλινε στη ίδια λύση για κάθε $\tau$. Όμως, η ταχύτητα σύγκλισης του αλγορίθμου ήταν διαφορετική για κάθε $\tau$. Παρατηρήθηκε ότι χρειάστηκαν λιγότερες επαναλήψεις (23) για $\tau =1.4$ ενώ για τις ακραίες τιμές, χρειάστηκαν ως και 326 επαναλήψεις. 

Για άλλα ζεύγη $A$ και $d$ υπήρχε ανάλογη συμπεριφορά με τη διαφορά ότι για μεγάλα $\tau$, ο αλγόριθμος δεν συνέκλινε. Επομένως, μπορεί να εξαχθεί το συμπέρασμα ότι η παράμετρος $\tau$ επηρεάζει αφενός τη σύγκλιση του αλγορίθμου και αφετέρου την ταχύτητά της.

Θεωρητική ανάλυση της σύγκλισης ή μη καθώς και της ταχύτητας μπορεί να γίνει υπολογίζοντας την φασματική ακτίνα του $G$, η οποία σε περίπτωση σύγκλισης πρέπει να είναι $\rho(G)<1$.

Παρακάτω ακολουθεί ο κώδικας που γράφτηκε σε \textlatin{Python} και έγινε χρήση της βιβλιοθήκης \textlatin{Numpy}. Οι ρουτίνες υπολογισμού δίνονται στο Παράρτημα.
\selectlanguage{english}
\lstinputlisting[style=python, firstline=8]{ex6.py}

\end{document}