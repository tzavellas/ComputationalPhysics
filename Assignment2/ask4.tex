\documentclass[assignment2.tex]{subfiles}
\begin{document}

\section*{4η Άσκηση}
Ο πίνακας A παραγοντοποιείται σε \textlatin{LU} ως:
\begin{equation}
\left[
\begin{matrix}
1 & 1 & 2 \\
-1& 0 & 2 \\
3 & 2 & -1
\end{matrix}
\right]
=
\left[
\begin{matrix}
1 & 0 & 0 \\
-1& 1 & 0 \\
3 & -1 & 1
\end{matrix}
\right]
\left[
\begin{matrix}
1 & 1 & 2 \\
0 & 1 & 4 \\
0 & 0 & -3
\end{matrix}
\right]
\end{equation}

Κατόπιν, μπορεί να λυθεί το αρχικό σύστημα $Ax=b$ ως $Ly=b$ με εμπρός αντικατάσταση και στη συνέχεια λύνοντας $Ux=b$ με πίσω αντικατάσταση. Για το πρώτο σύστημα είναι
\begin{equation}
y = \left[
\begin{matrix}
1 \\ -2 \\ 3
\end{matrix}
\right]
\end{equation}
και για το δεύτερο
\begin{equation}
x = \left[
\begin{matrix}
1 \\ 2 \\ -1
\end{matrix}
\right]
\end{equation}

Παρακάτω ακολουθεί ο κώδικας που γράφτηκε σε \textlatin{Python} και έγινε χρήση της βιβλιοθήκης \textlatin{Numpy}. Οι ρουτίνες παραγοντοποίησης \textlatin{LU}, εμπρός και πίσω αντικατάστασης δίνονται στο Παράρτημα.
\selectlanguage{english}
\lstinputlisting[style=python, firstline=8]{ex4.py}

\end{document}