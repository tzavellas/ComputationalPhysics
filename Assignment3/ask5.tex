\documentclass[assignment3.tex]{subfiles}
\begin{document}

\section*{5η Άσκηση}
Τα σημεία της συνάρτησης είναι $\lbrace0.25, 0.5, 0.75\rbrace$, δηλαδή ισαπέχοντα. Για την προσέγγιση των παραγώγων πρώτης και δεύτερης τάξης, μπορεί να εφαρμοστούν οι τύποι διαφορών (\ref{eq:derivative_gen}) και (\ref{eq:sec_derivative_gen}) για $n=2$. Εν προκειμένω, είναι $h=0.25$ το βήμα και $x=x_1$ το σημείο υπολογισμού των παραγώγων.

\begin{equation}
f'(x)\approx \frac{1}{h}\left[\Delta f_0 + \frac{1}{2}(2\theta -1)\Delta^2f_0+\cdots + \frac{d}{d\theta}\binom{\theta}{n}\Delta^nf_0\right]
\label{eq:derivative_gen}
\end{equation}

\begin{equation}
f''(x)\approx \frac{1}{h^2}\left[\Delta^2f_0 + (\theta -1)\Delta^3f_0+\cdots + \frac{d^2}{d\theta^2}\binom{\theta}{n}\Delta^nf_0\right]
\label{eq:sec_derivative_gen}
\end{equation}

Τελικά οι τύποι της πρώτης και δεύτερης παραγώγου δίνονται από τις εξισώσεις (\ref{eq:actual_derivatives}) και μπορούν εύκολα να γραφεί πρόγραμμα σε υπολογιστή.

\begin{equation}
\begin{split}
f'(x)&\approx\frac{1}{h}\left(\Delta f_0 + \frac{1}{2}(2\theta -1)\Delta^2f_0\right)\\
f''(x)&\approx \frac{1}{h^2}\Delta^2f_0
\end{split}
\label{eq:actual_derivatives}
\end{equation}

Μετά από εκτέλεση του προγράμματος προκύπτουν τα νούμερα του Πίνακα \ref{table:derivative_comparison}. Καταρχάς, οι πραγματικές τιμές της πρώτης και της δεύτερης παραγώγου διαφέρουν μόνο ως προς το πρόσημο, λόγω του ότι η $f(x)$ είναι η εκθετική συνάρτηση. Το σφάλμα είναι γενικά μικρό αλλά μπορεί να εκτιμηθεί και το άνω φράγμα του.

\begin{table}[ht]
	\centering
	\begin{tabular}{||c c c||} 
		\hline
		& Πραγματική & Αριθμητική \\ [0.5ex] 
		\hline\hline
		$f'(0.5)$ & -0.6065 & -0.6128 \\ 
		\hline
		$f''(0.5)$ & 0.6065 & 0.6096 \\ [1ex] 
		\hline
	\end{tabular}
	\caption{Σύγκριση τιμών παραγώγων}
	\label{table:derivative_comparison}
\end{table}

Από θεωρία, είναι γνωστή η σχέση (\ref{eq:error_derivative}), που προέρχεται από την αντίστοιχη σχέση σφάλματος στην παρεμβολή πολυωνύμων και μπορεί να ληφθεί η απόλυτη τιμή της.
\begin{equation}
E_n(x_k) = \prod_{i=0,i\neq k}^{n}(x_k-x_i)\frac{f^{(n+1)\xi}}{(n+1)!}
\label{eq:error_derivative}
\end{equation}

Ο τελικός υπολογισμός για το φράγμα της πρώτης παραγώγου δίνεται από την (\ref{eq:bound_error_derivative}). Το άνω φράγμα της $|e^{-\xi}|$ προέκυψε $e^{-0.25}$ γιατί η συνάρτηση είναι γνησίως φθίνουσα και στο διάστημα $(0.25, 0.75)$ θα είναι πάντα $|e^{-\xi}| < e^{-0.25}$.
\begin{equation}
\begin{split}
|E_2(0.25)| &= |x_1-x_0||x_1-x_2|\frac{|e^{-\xi}|}{3!}\rightarrow \\
& < \frac{e^{-0.25}|x_1-x_0||x_1-x_2|}{6} \rightarrow \\
&\approx 0.008
\end{split}
\label{eq:bound_error_derivative}
\end{equation}

Για την δεύτερη παράγωγο, ο αντίστοιχος τύπος του (\ref{eq:error_derivative}) είναι πολύπλοκος. Ωστόσο, μπορεί να παρατηρηθεί ότι, η $f''(x)=f(x)$ και επομένως σε αυτή την περίπτωση θα ισχύει ο τύπος σφάλματος παρεμβολής (\ref{eq:bound_error_sec_derivative}). Λαμβάνοντας την απόλυτη τιμή του τελευταίου υπολογίζεται η (\ref{eq:bound_error_sec_derivative2}).

\begin{equation}
f(x)-p_2(x)=\frac{(x-x_0)(x-x_1)(x-x_2)}{(3)!}f^{(3)}(\xi)
\label{eq:bound_error_sec_derivative}
\end{equation}

\begin{equation}
\begin{split}
|f(x)-p_2(x)|&=\left|\frac{(x-0.25)(x-0.5)(x-0.75)}{3!}e^{-\xi}\right| \rightarrow \\
&< \frac{|x-0.25||x-0.5||x-0.75|}{6}e^{-0.25}\rightarrow \\
& < \frac{\dfrac{1}{2} \dfrac{1}{2} \dfrac{1}{4}}{6} e^{-0.25}\rightarrow \\
&=\frac{e^{-0.25}}{96} \approx 0.008
\end{split}
\label{eq:bound_error_sec_derivative2}
\end{equation}

Μπορεί το φράγμα του σφάλματος και στις δύο περιπτώσεις να είναι περίπου ίδιο, αλλά υπάρχει ένα λεπτό σημείο. Στην περίπτωση της πρώτης παραγώγου, το φράγμα για το σφάλμα δίνεται για το $x=x_1$ μόνο, ενώ στην δεύτερη παράγωγο για όλο το διάστημα $[0.25, 0.75]$.

Παρακάτω ακολουθεί ο κώδικας που γράφτηκε σε \textlatin{Python} και έγινε χρήση της βιβλιοθήκης \textlatin{Numpy}. Οι ρουτίνες υπολογισμού εμπρός διαφορών και αριθμητικών παραγώγων δίνονται στο Παράρτημα.
\selectlanguage{english}
\lstinputlisting[style=python, firstline=8]{ex5.py}
\end{document}