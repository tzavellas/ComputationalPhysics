\documentclass[assignment3.tex]{subfiles}
\begin{document}

\section*{5η Άσκηση}
Τα σημεία της συνάρτησης είναι $0.25, 0.5, 0.75$, άρα ισαπέχοντα. Για την προσέγγιση των παραγώγων πρώτης και δεύτερης τάξης, μπορεί να εφαρμοστούν οι τύποι διαφορών (\ref{eq:derivative_gen}) και (\ref{eq:sec_derivative_gen}) για $n=2$. Εν προκειμένω είναι $h=0.25$ το βήμα και $x=x_1$ το σημείο υπολογισμού των παραγώγων.

\begin{equation}
f'(x)\approx \frac{1}{h}\left[\Delta f_0 + \frac{1}{2}(2\theta -1)\Delta^2f_0+\cdots + \frac{d}{d\theta}\binom{\theta}{n}\Delta^nf_0\right]
\label{eq:derivative_gen}
\end{equation}

\begin{equation}
f''(x)\approx \frac{1}{h^2}\left[\Delta^2f_0 + (\theta -1)\Delta^3f_0+\cdots + \frac{d^2}{d\theta^2}\binom{\theta}{n}\Delta^nf_0\right]
\label{eq:sec_derivative_gen}
\end{equation}

Τελικά οι τύποι της πρώτης και δεύτερης παραγώγου δίνονται από τις εξισώσεις (\ref{eq:actual_derivatives}) και μπορούν εύκολα να υλοποιηθούν σε πρόγραμμα.

\begin{equation}
\begin{split}
f'(x)&\approx\frac{1}{h}\left(\Delta f_0 + \frac{1}{2}(2\theta -1)\Delta^2f_0\right)\\
f''(x)&\approx \frac{1}{h^2}\Delta^2f_0
\end{split}
\label{eq:actual_derivatives}
\end{equation}

Μετά από εκτέλεση του προγράμματος προκύπτουν τα νούμερα του Πίνακα \ref{table:derivative_comparison}. Καταρχάς, οι πραγματικές τιμές διαφέρουν μόνο ως προς το πρόσημο, λόγω του ότι η $f(x)$ είναι η εκθετική συνάρτηση. Το σφάλμα είναι γενικά μικρό αλλά μπορεί να εκτιμηθεί και το άνω φράγμα του.

\begin{table}[ht]
	\centering
	\begin{tabular}{||c c c||} 
		\hline
		& Πραγματική & Αριθμητική \\ [0.5ex] 
		\hline\hline
		$f'(0.5)$ & -0.6065 & -0.6128 \\ 
		\hline
		$f''(0.5)$ & 0.6065 & 0.6096 \\ [1ex] 
		\hline
	\end{tabular}
	\caption{Σύγκριση τιμών παραγώγων}
	\label{table:derivative_comparison}
\end{table}

Από θεωρία, είναι γνωστή η σχέση (\ref{eq:error_derivative}), που προέρχεται από την αντίστοιχη σχέση σφάλματος στην παρεμβολή πολυωνύμων και μπορεί να ληφθεί η απόλυτη τιμή της.
\begin{equation}
E_n = \prod_{i=0,i\neq k}^{n}(x_k-x_i)\frac{f^{(n+1)\xi}}{(n+1)!}
\label{eq:error_derivative}
\end{equation}

Ο τελικός υπολογισμός για το φράγμα της πρώτης παραγώγου δίνεται από την (\ref{eq:bound_error_derivative}).
\begin{equation}
\begin{split}
|E_2| &= |x_1-x_0||x_1-x_2|\frac{|e^{-\xi}|}{3!}\rightarrow \\
& \leq \frac{e^{-0.25}|x_1-x_0||x_1-x_2|}{6} \rightarrow \\
&\approx 0.00811250
\end{split}
\label{eq:bound_error_derivative}
\end{equation}

Παρακάτω ακολουθεί ο κώδικας που γράφτηκε σε \textlatin{Python} και έγινε χρήση της βιβλιοθήκης \textlatin{Numpy}. Οι ρουτίνες υπολογισμού εμπρός διαφορών και παραγώγων δίνονται στο Παράρτημα.
\selectlanguage{english}
\lstinputlisting[style=python, firstline=8]{ex5.py}
\end{document}