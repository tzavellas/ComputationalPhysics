\documentclass[assignment1.tex]{subfiles}
\begin{document}

\section*{2η Άσκηση}

Η άσκηση απαιτεί υπολογισμό του ολοκληρώματος \ref{eq:moment}, κάνοντας χρήση της απλής μεθόδου \textlatin{Monte Carlo}. Το αποτέλεσμα του ολοκληρώματος έγινε αριθμητικά με χρήση του πακέτου \textlatin{Mathematica}, και προέκυψε ίσο με $I=1.72\cdot10^7$. Η μεγάλη τιμή του ολοκληρώματος είναι αναμενόμενη, καθώς η ολοκληρωτέα συνάρτηση αυξάνεται με μεγάλο ρυθμό στο διάστημα ολοκλήρωσης.

\begin{equation}
I=\int_0^L r^2\left(1+e^{r^2}\right) \mathrm{d}r
\label{eq:moment}
\end{equation}


Ο υπολογισμός \textlatin{Monte Carlo} παρήγαγε αποτέλεσμα $I=(17.2 \pm 0.2)\cdot 10^6$, που είναι πολύ κοντά στο αριθμητικό αποτέλεσμα. Επιπλέον, μολονότι το απόλυτο σφάλμα φαίνεται μεγάλο, το σχετικό σφάλμα είναι μόνο $1.16\%$, άρα μικρό.

Αν γίνει ολοκλήρωση σε μέρη, αναμένεται η ίδια τιμή ολοκληρώματος αλλά μικρότερο σφάλμα. Αυτό οφείλεται στο γεγονός ότι το ολοκλήρωμα κάθε μέρους αποτελεί στην ουσία μια μέτρηση στην οποία αντιστοιχεί ένα σφάλμα. Η μέτρηση δίνεται από την εξίσωση (\ref{eq:part_integral}) και το σφάλμα από την εξίσωη (\ref{eq:part_error}). 

\begin{equation}
I_i = (x_{i+1}-x_i) \sum_{i=1}^{N_i} f(r_i) 
\label{eq:part_integral}
\end{equation}

\begin{equation}
\delta I_i = \frac{x_{i+1}-x_i}{\sqrt{N_i}} \sqrt{\frac{1}{N_i}\sum_{i=1}^{N_i} f^2(r_i) - \left( \frac{1}{N_i}\sum_{i=1}^{N_i} f(r_i) \right)^2}
\label{eq:part_error}
\end{equation}

Αν το τελικό αποτέλεσμα είναι $I=\sum_{i=1}^{M} I_i$, τότε το σφάλμα του είναι $\delta I = \sqrt{\sum_{i=1}^{M}\delta I_i^2}$, σύμφωνα με το νόμο διάδοσης σφαλμάτων. Επομένως, το σφάλμα του ολοκληρώματος θα οφείλεται κυρίως στο μέρος των μετρήσεων που αντιστοιχούν στο πιο μεγάλο $\delta I_i$, και όχι ισομερώς σε όλες τις μετρήσεις όπως στην απλή περίπτωση \textlatin{Monte Carlo}.


Πράγματι για ολοκλήρωση σε 4 μέρη, προκύπτει $I=(17.20 \pm 0.19)\cdot 10^6$, δηλαδή σχετικό σφάλμα $1.10\%$. Τέλος, για ολοκλήρωση σε 16 μέρη προκύπτει $I=(17.20 \pm 0.11)\cdot 10^6$ και σχετικό σφάλμα $0.64\%$. Παρατηρείται ότι, η αύξηση του αριθμού των μερών οδηγεί σε μείωση του σφάλματος της μεθόδου, χωρίς υπολογιστική επιβάρυνση, μιας και εκτελείται περίπου ο ίδιος αριθμός πράξεων κινητής υποδιαστολής.

Παρακάτω ακολουθεί ο κώδικας που γράφτηκε σε \textlatin{Python} και έγινε χρήση της βιβλιοθήκης \textlatin{Numpy}.

\selectlanguage{english}
\lstinputlisting[style=python, firstline=8]{mc_MomentOfInertia.py}

\end{document}