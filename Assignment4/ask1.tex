\documentclass[assignment4.tex]{subfiles}
\begin{document}

\section*{1η Άσκηση}
Ζητείται η ροπή αδράνειας της ράβδου με πυκνότητα $\rho(x) = 1+x^3$ με $x\in(0,3)$ ως προς άξονα που περνάει από το $x=0$. Αναλυτικά, η ροπή αδράνειας υπολογίζεται από την εξίσωση (\ref{eq:analytic_moment}).
\begin{equation}
\begin{split}
I &= \int_{0}^{3} x^2\rho(x) dx \rightarrow \\
&= \int_{0}^{3} x^2 + x^5 dx \rightarrow \\
&= 130.5
\end{split}
\label{eq:analytic_moment}
\end{equation}

Το ολοκλήρωμα υπολογίστηκε με τη μέθοδο του τραπεζίου καθώς και με τη μέθοδο \textlatin{Simpson}. Η μέθοδος του τραπεζίου έδωσε $I_{tr}=130.5012$ και η μέθοδος \textlatin{Simpson} $I_{Sim}=130.5002$. Η απόκλιση της πρώτης είναι $\delta I_{tr}\approx0.0012$ και της δεύτερης είναι $\delta I_{Sim}\approx0.0002$. 

Και οι δύο μέθοδοι έδωσαν αποτέλεσμα με τη ζητούμενη ακρίβεια ($10^{-2}$), ωστόσο, στη μέθοδο τραπεζίου απαιτήθηκαν $n=10$ επαναλήψεις, ενώ στην \textlatin{Simpson} $n=6$, δηλαδή σχεδόν οι μισές. Επιπλέον, η μέθοδος \textlatin{Simpson} είχε μικρότερη απόκλιση από το πραγματικό αποτέλεσμα.

Στη μέθοδο του τραπεζίου, σε κάθε νέα επανάληψη, ο αριθμός των τραπεζίων διπλασιάζεται και επομένως κάποια από τα σημεία της προηγούμενης επανάληψης είναι κοινά. Τελικά, το ολοκλήρωμα υπολογίζεται από την εξίσωση (\ref{eq:trapezoid_moment}) και επομένως για $n=10$ επαναλήψεις απαιτούνται $N=2+n-1=11$ υπολογισμοί της $\rho(x)$. 

Σε μια έξυπνη και αποδοτική υλοποίηση του αλγορίθμου, οι τιμές της $\rho(x)$ μιας επανάληψης, αποθηκεύονται για χρήση στην επόμενη επανάληψη. Στη συγκεκριμένη όμως υλοποίηση, δεν ήταν απαραίτητη αυτή η βελτιστοποίηση και έτσι στην $i+1$ επανάληψη απαιτούνται $Ν_{i+1} = 2Ν_i + 1$ υπολογισμοί. Συνολικά, απαιτήθηκαν $Ν_{tot}=1024$ υπολογισμοί.
\begin{equation}
I \approxeq \frac{h}{2}\left[ f_0 + 2\sum_{i=1}^{n-1}f_i + f_n \right]
\label{eq:trapezoid_moment}
\end{equation}

Για τη μέθοδο \textlatin{Simpson}, σε κάθε υποδιάστημα χρησιμοποιούνται 3 σημεία και όμοια σε κάθε επανάληψη ο αριθμός των υποδιαστημάτων διπλασιάζεται. Το ολοκλήρωμα δίνεται από την εξίσωση (\ref{eq:simpson_moment}) και επομένως για $n=6$ επαναλήψεις απαιτούνται $N=2+\frac{n}{2}-1+\frac{n}{2}=7$ υπολογισμοί της $\rho(x)$.

Όπως και στην περίπτωση του τραπεζίου, η υλοποίηση του αλγόριθμου \textlatin{Simpson} δεν είναι η βέλτιστη για τους ίδιους λόγους και τελικά απαιτήθηκαν $N_{tot}=64$ υπολογισμοί, δηλαδή ακριβώς οι μισοί σε σχέση με τη μέθοδο του τραπεζίου.

\begin{equation}
I \approxeq \frac{h}{3}\left[ f_0 + 2\sum_{i=1}^{\frac{n}{2}-1}f_{2i} + 4\sum_{i=1}^{n/2}f_{2i-1} + f_n \right]
\label{eq:simpson_moment}
\end{equation}

Συμπερασματικά, προκύπτει ότι η μέθοδος \textlatin{Simpson} είναι λίγο πιο σύνθετη στην υλοποίηση, αλλά υπολογιστικά πιο ``φθηνή'' από τη μέθοδο τραπεζίου, για την ίδια ακρίβεια.

Παρακάτω ακολουθεί ο κώδικας που γράφτηκε σε \textlatin{Python} και έγινε χρήση της βιβλιοθήκης \textlatin{Numpy}. Οι λεπτομέρειες υλοποίησης των μεθόδων τραπεζίου και \textlatin{Simpson} δίνονται στο Παράρτημα.
\selectlanguage{english}
\lstinputlisting[style=python, firstline=8]{ex1.py}
\end{document}